O problema do troco consiste em, dado um valor (o troco) a ser pago de volta para alguém, Minimizar a quantidade de notas e moedas a devolver.

O programa troco-iaa.c, listado na Listagem 1 - abaixo, resolve o problema do troco. O programa foi compilado e executado. O resultado da execução é apresentado na Listagem 2

Listagem 1 - troco-iaa.c

 1
 2
 3
 4
 5
 6
 7
 8
 9
10
11
12
13
14
15
16
17
18
19
20
21
22
23
24
25
26
27
28
29
30
31
32
33
34
35
36
37
38
39
40
41
/*troco-iaa.c*/
#include <stdio.h>

/* array que nas posicoes pares contem o valor da 
   moeda e nas posicoes impares contera a quantidade 
   de moedas a devolver. */
int moedas[]={ 
     1, 0,
     5, 0,
    10, 0,
    20, 0,
    50, 0,
    100, 0
};
// quantidade de diferentes valores de moedas
int nMoedas=6;

/* a funcao a analisar.
   Recebe como argumento o valor a ser devolvido e
   calcula a quantidade de moedas a devolver. */
void calculaQuantasMoedas (int v) {
  for (int i=nMoedas-1;i>=0;i--) {
    moedas[i*2+1]=v/moedas[i*2];
    v=v%moedas[i*2];
  }
}

/* funcao que imprime quantas moedas devem 
   ser devolvidas. */
void imprimeMoedas () {
  for (int i=0;i<nMoedas;i++) {
    printf ("%d moedas de valor %d.\n", 
      moedas[i*2+1], moedas[i*2]);
  }
}

void main () {
  int valor=235;
  calculaQuantasMoedas (valor);
  imprimeMoedas ();
}
HTML generated using hilite.me



Listagem 2: compilação e execução de troco-iaa.c

fabio@super:~/IAA$ gcc troco-iaa.c 
fabio@super:~/IAA$ ./a.out
0 moedas de valor 1.
1 moedas de valor 5.
1 moedas de valor 10.
1 moedas de valor 20.
0 moedas de valor 50.
2 moedas de valor 100.
A função de interesse (pois é a que, de fato, faz as contas) é calculaQuantasMoedas (int v)

Qual variável/Característica do programa você indica como o tamanho do problema? Justifique.
Dada sua escolha, diga qual a complexidade de tempo da função calculaQuantasMoedas (int v) , Justifique.
O elementos do array de moedas podem estar em qualquer ordem? Justifique. 








Texto de resposta Questão 1
Análise do Problema do Troco


1. Indicação do tamanho do problema
. A variável que melhor representa o tamanho do problema é nMoedas
. Isso ocorre porque nMoedas
 define a quantidade de diferentes tipos de moedas que o algoritmo precisa processar. O número de divisões e operações de módulo realizadas dentro do loop for
 depende diretamente do valor de nMoedas
. Portanto, quanto maior o valor de nMoedas
, mais operações são necessárias para calcular a quantidade de moedas a serem devolvidas, o que aumenta proporcionalmente o tamanho do problema.


2. Complexidade de tempo
A complexidade de tempo da função calculaQuantasMoedas
 é O(n)
, onde n
 representa o número de tipos de moedas. 
Demonstração:
- O loop for
 itera de nMoedas−1
 a 0, ou seja, realiza a iteração de nMoedas
. Em cada iteração, duas operações principais são realizadas: divisão e módulo, ambas com complexidade constante O(1)
. 

- Como o loop executa nMoedas
 vezes e a complexidade de cada iteração é O(1)
, a complexidade total da função é O(n)
, onde n
 é o número de tipos de moedas. 

- A função pode ser representada pela fórmula f(n)=c⋅n
, onde c
 é uma constante que representa o tempo necessário para cada iteração do loop, e n
 é o número de tipos de moedas.  


A função pertence a O(n)
 devido: 

De acordo com a definição O(g(n)):{f:0≤f(n)≤c∗g(n)}
 
0≤c1∗n≤c∗n

Seleção de valores para c, c1 e n0: c=c1 n0 = 1 c1∗n≤c∗n

Esta desigualdade vale para todo n≥n0
. 

Isso prova que f(n)
 pertence a O(n)



3. Pode estar em qualquer ordem?
Não, os elementos do array de moedas não podem estar em qualquer ordem; é necessário que estejam ordenados de forma decrescente. Isso se deve ao fato de que o algoritmo utilizado na função calculaQuantasMoedas
 é um algoritmo guloso, que itera sobre o array de moedas do maior para o menor valor. Dessa forma, ele sempre seleciona a moeda de maior valor que ainda pode ser utilizada no troco, garantindo a devolução do menor número possível de moedas. Se os elementos não estiverem ordenados, o algoritmo pode escolher uma moeda de menor valor antes de considerar uma de valor maior, o que resultaria em um número maior de moedas devolvidas ou até mesmo em uma combinação de moedas incorreta. Portanto, a eficácia do algoritmo depende diretamente da ordenação decrescente dos valores no array.
Questão 2
Completo
Vale 6,00 ponto(s).
Marcar questão
Texto da questão
Considere o algoritmo que apresenta a lista de movimentos que resolve o problema da torre de Hanoi apresentado em aula https://edisciplinas.usp.br/pluginfile.php/8500040/mod_resource/content/1/hanoi.c

Obter e apresentar a complexidade de tempo da função hanoi(char ori, char dst, char aux, int nro)

   Espera-se que o procedimento usado para responder a esta pergunta seja : 1) Avaliar a complexidade de tempo de cada linha da função; 2) Calcular a complexidade de tempo da função (que, neste caso, costuma ser obtida como uma fórmula de recorrência); 3) Obter a fórmula fechada equivalente à fórmula de recorrência; 4) Aplicar a definição dos operadores assintóticos usando um deles para notar a complexidade de tempo da função. 
    A presença/ausência de cada ítem do procedimento e da respectiva justificativa (ex. demonstração por indução no ítem 3 e aplicação da definição do(s) operador(es) no ítem 4) resultará na nota atribuída à resposta apresentada. (ié não vale copiar a resposta da wikipedia...)



Texto de resposta Questão 2
Demonstração da Complexidade de Tempo da Função Hanoi
1. Avaliação da complexidade de tempo de cada linha da função
Considerando o código recursivo para a função Hanoi:

O caso base if(nro == 1) realiza um número constante de operações printf, o que tem complexidade O(1).
Na parte recursiva, a função faz três chamadas recursivas dependendo de parâmetros. A complexidade total das chamadas recursivas depende do valor de nro por conta disso.
2. Cálculo da complexidade de tempo como uma fórmula de recorrência
Para determinar a complexidade total da função, é necessário analisar como o tempo de execução é afetado pelas chamadas recursivas. A fórmula de recorrência para a função é:

T(n)={O(1)2⋅T(n−1)+O(1)se n=1se n>1

O caso base n = 1 executa um número constante de operações \( O(1)) \).
Para n > 1, a função faz duas chamadas recursivas para \( n-1 \) discos e realiza operações adicionais constantes. Portanto, a fórmula é \( 2 \cdot T(n-1) + O(1) \).
Melhorando com Hipótese de Indução:

Supondo que a fórmula seja verdadeira para um número de discos \( n = k \), devemos provar que a fórmula também se aplica para k + 1 (passo de indução). Portanto:

T(k+1)=2⋅T(k)+c

Agora, considerando \( n = k + 1 \), cada chamada recursiva da função:

Primeira chamada: A função é chamada com \( n = k \) e deve mover \( k \) discos da torre de origem para a torre auxiliar. O tempo necessário para isso é T(k).
Segunda chamada: Em seguida, o disco de maior valor (o k+1-ésimo) será movido da torre de origem para a torre de destino, o que leva um tempo constante c.
Terceira chamada: Finalmente, a função é chamada novamente com n = k para mover os k discos da torre auxiliar para a torre de destino, o que também leva um tempo T(k).
Portanto, o tempo total é dado pela soma dos tempos das chamadas recursivas e do tempo para mover o maior disco:

T(k+1)=T(k)+c+T(k)=2⋅T(k)+c

3. Obtenção da fórmula fechada da recorrência
Para resolver a fórmula de recorrência \( T(n) = 2 \cdot T(n-1) + O(1) \), podemos expandir a fórmula:

T(n)=2⋅T(n−1)+O(1)
 T(n)=2⋅[2⋅T(n−2)+O(1)]+O(1)
 T(n)=22⋅T(n−2)+2⋅O(1)+O(1)

Generalizando para k iterações:

T(n)=2k⋅T(n−k)+∑k−1i=02i⋅O(1)

Quando \( k = n-1 \), a fórmula torna-se:

T(n)=2n−1⋅T(1)+∑n−2i=02i⋅O(1)

Como \( T(1) = O(1) \) e a soma da progressão geométrica é \( 2^n - 1 \):

T(n)=2n−1⋅O(1)+O(2n−1)
 T(n)=O(2n)

4. Aplicação da definição dos operadores assintóticos
Para mostrar que \( T[(n) \) pertence a \( O(2^n) \), usando a definição formal de notação O:

O(g(n))=f:0≤f(n)≤c⋅g(n) para algum c e para todo n≥n0

Aplicando:

O(2n)=f:0≤f(n)≤c⋅2n para algum c e para todo n≥n0

Com f(n) = T(n):

0≤T(n)≤c⋅2n

Escolha dos valores para \( c \) e \( n_0 \):

Escolhendo c = 1 (ou qualquer valor maior, se necessário) e \( n_0 = 1 \);
Assim, a desigualdade será: \( 0 \leq T(n) \leq 2^n \)
Esta desigualdade é verdadeira para todo \( n \geq n_0 \), provando que T(n) está em \( O(2^n) \).